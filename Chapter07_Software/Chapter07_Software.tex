\section{Software implementation -- \texttt{pyunicorn} }\label{sec:Software}

    %  JFD: Could be further shortened if needed.

	In this chapter, we briefly introduce the Python software package \texttt{pyunicorn}, which implements methods from both complex network theory and nonlinear time series analysis, and unites these approaches in a performant, modular and flexible way \cite{Donges2015}. Here, we mainly present a brief introduction of \texttt{pyunicorn} and a discussion of software structure and related computational issues. More details of the illustrative examples have been presented in \cite{Donges2015}. Although in the tutorial of \cite{Donges2015}, the work flow of using \texttt{pyunicorn} is mainly illustrated drawing upon examples from climatology, we have to emphasize that the package is applicable to all fields of study where the analysis of (big) time series data is of interest, e.g. in neuroscience \cite{Bullmore2009,Subramaniyam2014,Subramaniyam2015}, hydrology \cite{sun2015global} or economics and finance \cite{Wang2012}.

	\texttt{pyunicorn} is intended to serve as an integrated container for a host of conceptionally related methods which have been developed and applied by the involved research groups for many years. Its aim is to establish a shared infrastructure for scientific data analysis by means of complex networks and nonlinear time series analysis and it has already greatly taken advantage from the backflow contributed by users all over the world. The code base has been fully open sourced under the BSD 3-Clause license.

	First of all, we emphasize that \texttt{pyunicorn} covers rather general topics of complex network studies. The \texttt{pyunicorn} library consists of five subpackages: (1) {\it core}, which contains the basic building blocks for general network analysis and modeling. For instance, it is capable for analyzing and modeling general complex networks, spatial networks, networks of interacting networks or multiplex networks and node-weighted networks. (2) {\it funcnet}, which contains advanced tools for construction and analysis of general functional networks \cite{Bullmore2009,Donges2012PhD}. For instance, this module calculates cross-correlation, mutual information, mutual sorting information and their respective surrogates for large arrays of scalar time series. (3) {\it climate}, which focus on the construction and analysis of climate network \cite{Tsonis2004,Yamasaki2008,Donges2009,Donges2009b} and coupled climate network analysis \cite{Donges2011b}. (4) {\it timeseries}, which provides various tools for the analysis of non-linear dynamical systems and uni- and multivariate time series. This subpackage covers all aspects of RNs (Section \ref{sec:RecurrenceNt}) and (H)VGs (Section \ref{sec:VisibilityGt}) that have been reviewed in this report, except ordinal pattern transition networks. Furthermore, \texttt{pyunicorn} also presents methods for generating surrogate time series \cite{Schreiber2000}, which are useful for both functional networks and network-based time series analysis. (5) {\it utils}, which includes \texttt{MPI} parallelization support and an experimental interactive network navigator.

	\texttt{pyunicorn} is conveniently applicable to research domains in science and society as different as neuroscience, infrastructure and climatology. Most computationally demanding algorithms are implemented in fast compiled languages on sparse data structures, allowing the performant analysis of large networks and time series data sets. The software's modular and object-oriented architecture enables the flexible and parsimonious combination of data structures, methods and algorithms from different fields. For example, combining complex network theory and RPs yields RN analysis (Section \ref{sec:RecurrenceNt}) \cite{Donges2015}.

	Along these lines, \texttt{pyunicorn} has the potential to facilitate future methodological developments in the fields of network theory, nonlinear time series analysis and complex systems science by synthesizing existing elements and by adding new methods and classes that interact with or build upon preexisting ones. Nonetheless, we urge users of the software to ensure that such developments are theoretically well-founded as well as motivated by well-posed and relevant research questions to produce high-quality research.

	Besides \texttt{pyunicorn}, some other software packages are available, for instance, the MATLAB toolbox \texttt{CRP Toolbox} that allows to get the adjacency matrix of a RN (i.e., the recurrence matrix). Furthermore, this toolbox provides the computation of several network measures as well, for instance, the node degree $k$ (corresponds to the recurrence rate), the clustering coefficients $\mathcal{C}$ and transitivity $\mathcal{T}$. Some basic elements of recurrence network analysis have been published as Mathematica demonstrations \cite{Zech2010a,Zech2010b}. While we are not aware of further existing comprehensive software packages in the style of \texttt{pyunicorn} or \texttt{CRP Toolbox}, other groups have sometimes published code implementing specific methods of network-based nonlinear time series analysis, such as for example visibility graph analysis developed by the group of Lacasa et al. (\url{http://www.maths.qmul.ac.uk/~lacasa/Software.html}). In addition, we recommend some network visualization toolboxes, including Gephi (\url{http://gephi.org}) or Networkx \url{http://networkx.github.io} which can be used to analyze the networks once the adjacency matrix is available.
