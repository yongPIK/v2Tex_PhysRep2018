\section{Conclusions and future perspectives} \label{sec:Discussion}

	\subsection{Conclusions}
Time series analysis by means of complex networks is an innovative and powerful
approach with many ramifications and applications. In this report, we have reviewed several major algorithms for transforming a time series into a network representation, depending on the definitions of vertices and edges. The network approach makes use of different established methods, such as Markov chains or recurrences, but also of more abstract concepts, as visibility graphs, which form three main classes of methods, namely, recurrence network (RN), visibility graphs (VG) and transition network (TN) that have been discussed in detail throughout this report.


These methods complement available approaches with alternative measures, e.g., describing geometrical properties of the system under study in its phase space, but also broadening the applicability of time series analysis to short, complex, and multivariate data. As such, network based time series analysis can be used to characterize systems dynamics from a single time series, to distinguish different dynamics, to identify regime shifts and dynamical transitions, to test for time series reversibility, or to predict the future system states.

%\todo[inline]{
%
%Comments by Jonathan:
%
%- What is the added value of network methods compared to standard methods?
%
%- Why do network methods typically seem to work better with small data sets than standard methods? Does this have to do with $N^2$ of information units (although not independent) being created from N data points through building an adjacency matrix, leading to more robust statistics?
%
%- What are common applications of network methods? Such as identifying regime shifts / transitions / tipping points in time series, distinguishing different dynamical regimes, analysis of dynamical system structure in phase space, prediction of some types... What more?
%
%We might probably start with taking these points already in the introduction (as motivating questions beyond the methodological interest) and take them up for more detailed discussion at the end of the review.}

	\subsection{Future perspectives}
	As shown in this review, applying complex network approaches in the context of time series analysis has already gained a number of valuable insights from both, a theoretical dynamical systems and/or stochastic processes perspective and in the context of various types of applications. However, as for any emerging field, there is a large body of relevant questions and upcoming developments that may further increase the relevance of the discussed frameworks. For example, in the particular case of RNs, there are some evident questions, the most relevant being about the invariance of findings under variation of the threshold value $\varepsilon$ since this value determines the link density of the network and all network characteristics become trivial in the limit of full connectivity \cite{Bradley2015c}. Therefore, future work needs to further demonstrate the wide applicability of existing as well as newly developed methods for transforming time series into complex networks by considering additional applications from various disciplines, particularly regarding the methods' capabilities to provide deeper insights beyond the already existing knowledge of the respective topics. Depending on the particular working subject, it will be crucial to use network analysis to extract some features that are not easily captured by most of standard methods of linear and nonlinear time series analysis, thereby demonstrating the added value of the network methods.

	To this end, we would like to highlight some particularly prospective directions for future research, being aware of the fact that this selection will be necessarily incomplete and subjective.

\subsubsection{Evolving and temporal network analysis for time series}

In their basic formulations as discussed in this review, most existing approaches for transforming and analyzing time series from a complex network perspective have been designed primarily to cope with stationary systems. However, generalizations are desirable that account for the fact that real-world time series often exhibit changing dynamical patterns as a hallmark of nonstationarity.

While many complex network approaches traditionally assume static network structures, there are natural extensions of evolving network analysis, i.e., the consideration of complex networks that change as a function of time. For proximity networks as well as visibility graphs and related concepts, a generalization to evolving networks appears straightforward if we consider both node and edge sets being time-dependent. In turn, for transition networks, one may keep the underlying node set but consider the transition frequencies between patterns encoded in the weights of the directed edges as changing with time.

In this context, the most common way to generate evolving time series networks would be employing a sliding windows analysis. Here, evolving networks can be understood as successive snapshots of static networks obtained for individual, mutually overlapping time windows. As a result, we may trace changes in the resulting network properties over time and use them as proxies for dynamical changes in the underlying time series, thereby revealing non-stationarity of the system or even distinct episodic events such as regime shifts. However, the sliding windows approach brings about some natural limitation, that is, the necessity of making empirical choices for the temporal window lengths and the mutual window overlap, for which there are no optimal strategies but rather heuristics depending on the individual case study. In general, using extremely small window sizes between two consecutive snapshot networks allows for a high resolution of tracked changes in the network properties, but could obscure slower trends which only become visible over longer time-scales. Conversely, using larger window sizes integrates information over considerably large time intervals and thus loses both, resolution and information on the effects of individual events within each window. Therefore, objective strategies for choosing an appropriate time-scale for dividing the evolution of a network into static snapshots are required, which are likely to have positive effects on proper interpretations of the obtained results \cite{Donges2011,Zou2014,schleussner2015indications,Franke2017}.

Even with such optimal and objective choices of time windows, a sliding window technique as described above by definition cannot cover all potentially relevant aspects associated with the temporal structures in the underlying time series that should be captured by their network representations \cite{Holme2012}. For this purpose, we need to include an additional time dimension to take the detailed information on the temporal succession of network structures (emergence and/or disappearance of nodes and links) into account in the context of quantitative analyses. In this spirit, there have been attempts to analyse visibility graphs as temporal networks \cite{Mutua2015}. Furthermore, there might be cases in transition networks where transitions between patterns do exist at certain times but not at others, i.e., times with active versus inactive links (so-called blinking links \cite{Gozolchiani2008}). In such a situation, the time ordering of observations in the underlying time series can have important effects that cannot be captured by static network representations. In the context of time series analysis, Weng {\textit{et al.}} \cite{Weng2017} proposed to transform time series into temporal networks \cite{Holme2012} by encoding temporal information into an additional topological dimension of the graph, which captures the ``lifetime" of edges. We note that a proper modification of the ordinal pattern transition network approach (for instance, considering short-term transition networks) may provide the necessary temporal information for this problem since the transition matrix describes the probability of future evolution directions of the observed trajectory.

\subsubsection{Multilayer and multiplex network analysis for multiple time scale time series}

In this work, we have provided a review on existing methods for reconstructing multilayer and multiplex networks from time series, for instance, multiplex recurrence networks, multiplex visibility graphs, inter-system recurrence networks, and joint recurrence networks. However, most of these methods are only appropriate for stationary time series as we have discussed in Section~\ref{subsec:practicalRN}. When monitoring complex physical systems over time, one often finds multiple phenomena in the data that work on different time scales. For example, observations are collected on a minimal (short) time scale, but also reflect the time series' behavior over larger time scales, which is rather typical for real-world climate data. Another prominent example from neuroscience is the recording of spiking activity of individual neurons (discrete event series) and local field potentials (time continuous measurement). Higher-frequency variability of such data can obscure the time series behavior of the data at larger scales, making it more difficult to identify the associated trends. If one is interested in analyzing and modeling these individual phenomena, it is crucial to recognize the multiple time scales in the construction of multilayer and multiplex networks from time series. One corresponding way could be applying successive scale-sensitive filters prior to network generation, with the choice of a particular method depending on the specific data set and research question, such as empirical mode decomposition \cite{gao2018} or some type of wavelet transform \cite{chen2012}.

\subsubsection{The inverse problem of time series regeneration from networks}

		Most of the existing works focus on investigating proper transformation methods for mapping time series into network representations. To study the inverse question of how much information is encoded in a given network model of a time series \cite{Wiedermann2017}, some studies have been undertaken to recover the original time series from the network, to use the network to reconstruct the phase space topology of the original system, or to generate new time series from the networks and compare these with the original \cite{Campanharo2011,hirata2008,Hirata2016,McCullough2017}. This inverse problem of getting back from the network adjacency matrix to time series of the underlying dynamical system remains a big challenge, which certainly has many applications \cite{Lancaster2018}. In general, transformations of complex networks to time series are not straightforward. Specifically, without having additional node labels informing about the temporal succession of vertices, the order of vertices in a complex network can be arbitrarily exchanged without affecting the network topology. In turn, for reconstructing the trajectory from a network representation, the temporal order of the nodes needs to be known.

A few algorithms have been proposed so far to reconstruct time series from networks. For instance, under a certain condition for reconstructability, Thiel {\textit{et al.}} proposed an algorithm to reconstruct time series from their recurrence plots \cite{thiel2004b,Robinson2009}. In this case, the reconstructed attractor shows topological equivalence with the original attractor \cite{Zhao2014}. Furthermore, based on recurrence plots with fixed
number of recurrences per state (equivalent to $k$-nearest neighbor networks), the topological properties of the underlying time series have been reconstructed by multidimensional scaling \cite{hirata2008}. Recently, it has been shown that $k$-nearest neighbor and $\varepsilon$-recurrence networks can be viewed as identical structures under a change of (equivalent) metrics \cite{Khor2016}. Based on this fact, an improved inversion algorithm has been proposed in \cite{Khor2016}, which further supports the use of complex networks as a means of studying dynamical systems, while also revealing an equivalence between $\varepsilon$-recurrence and $k$-nearest neighbor classes of complex networks. In addition, algorithms based on random walks have been proposed in the literature. For instance, a random walk algorithm has been used in \cite{Hou2015}, which further compares the performance of RNs and adaptive $k$-nearest neighbor networks. The performances of these algorithms have been compared in \cite{Liu2013c}. Recently, a constrained random walk algorithm has been proposed to regenerate time series from ordinal transition networks \cite{McCullough2017}.

For all these different algorithms, there are several important algorithmic parameters that have to be chosen empirically in order to guarantee consistent topology between the reconstructed time series and the original system. The general performance and applicability of each algorithm has to be evaluated in future work. One application of such regeneration algorithms is to perform surrogate analysis, for example, to test for the statistical significance of the results obtained from analyzing the original time series. Therefore, we also have to take into account the proper choice of null hypothesis while proposing algorithms for regenerating time series from networks.

\subsubsection{Combining data mining tools with time series network approaches}
In the framework of time series mining \cite{FU2011}, some fundamental tasks include dimension reduction by introducing proper indexing mechanisms, similarity comparison between time series subsequences and segmentation. The final goal of mining tools is to discover hidden information or knowledge from either the original or the transformed time series, for instance, using a proper clustering method to identify patterns. Note that the interesting pattern to be discovered here relate to rather general categories, including patterns that appear frequently versus such that occur rather surprisingly in the datasets \cite{FU2011,Aghabozorgi2015}. From the perspective of time axis, time series clustering can be classified into three categories, whole time series clustering, subsequence clustering and time point clustering \cite{Aghabozorgi2015}. Several algorithms have been proposed to perform time series clustering based on shapes of raw time series, feature vectors of dimension reduced time series, and distances between parametric model outputs and raw time series. These conventional mining algorithms have found various applications to time series of different origins, which, however, are challenged by practical issues like high dimensionality, very high feature correlations, and large amount of noise.

Despite the rapid increase in size and complexity of datasets in the era of big data, a proper combination of data mining tools with the complex network approaches for time series analysis has largely remained untouched so far. Such a joint research effort should combine methodologies and techniques from different fields, such as statistics, data mining, machine learning and visualization. It has been recently demonstrated that complex network approaches and data mining tools can indeed be integrated to provide novel insights for the understanding of complex systems \cite{Zanin2016}. From the viewpoint of nonlinear time series analysis, both sides of data mining and nonlinear time series analysis can benefit from each other, which will be one important topic for future research.

\subsubsection{Building network models for time series prediction}

Time series modeling and forecasting has attracted a great number of researchers' attention and provides the core of nonlinear time series analysis \cite{kantz1997,Bradley2015c}. To this end, we can build a proper model to forecast the system's future behavior, given a sequence of observations of one or a few time variable characteristics. Most existing methods originating from nonlinear dynamics are state-space models, which build local models in ``patches'' of a reconstructed state space and then use these models to predict the next point on the system's trajectory, which remains an active area of research \cite{Bradley2015c}. We note that most existing network approaches to nonlinear time series analysis have been focusing on characterizing network features of phase space (and, hence, diagnosing rather than forecasting the observed dynamics). Examples for using network approaches for time series modeling and prediction have not been reported in the literature yet to out best knowledge, but could provide another exciting future research avenue.
